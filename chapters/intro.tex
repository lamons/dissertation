\chapter{Introduction}
\begin{chapquote}[3em]{T.\ S.\ Eliot, \textit{\addfontfeature{ItalicFeatures={Style=Swash}}The Rock}}
	{\fancyfont
		{\addfontfeature{ItalicFeatures={Style=Swash}}Where} is the Life we have lost in living?\\
		\hskip 4.5em Where is the wisdom we have lost in knowledge?\\
		\hskip 3em Where is the knowledge we have lost in information?}
\end{chapquote}

\section{Background}
\dcap{I}{n the summer} of 2016, when Xu Qiuxia, a journalist from Hong Kong, was giving a lecture about journalism to a group of high school students in Beijing, she surprisingly noticed that when she was showing them a 16 years old piece of video footage of the ex-president of China Jiang Zemin, most of the students could recite the whole conversation in the video from their memory. While she didn't know at that time, the same video had been uploaded to the Internet, and already got 1,690,000 views\footnote{Until 15th Aug 2017, there were already 3,019,778 views. The referred video is available at \url{https://youtu.be/lBOwxXxesBo}, was uploaded to YouTube on 3rd Apr 2007.} on YouTube at that time \citep{__2016-2}. The culture phenomena called \moha\ became increasingly popular on Chinese Internet, memes, quotes, and derivatives of Jiang Zemin were created and circulated all over the Internet. Although these memes and their materials have been repeatedly censored and blocked, they have not disappeared, but constantly reappear in more and more abstract and implicit forms.

Building on theories from fields of urban design, politics, and media study, this dissertation intends to build a theoretical framework around two hypotheses. First, space is understood as the medium of political activities, means of surveillance, and the arena where the dominating and subordinate classes maintain or challenge the established social order. While the opportunity of open, organized and direct protest is restricted for lacking economic ability or political power, the subordinate class would tend to use everyday forms of resistance. This entails everyday activities and language occurring in the public space that are creatively hijacked for political purposes to express the discontents and demands of subordinates. Second, the virtual space that is maintained by the Information Communication Technologies (\ict s) could be considered as an extension, in some occasions an alternative of the physical space. The technique of \ict s have several advantages such as synchronicity and decentralization, which provide a chance to create alternative public ``space'', providing new ways of communication and acceleration the existing connections. However, due to complex socio-economical factors in the reality such as language and the economic ability of setting up Internet infrastructure, the social structure and regulative power practiced by the state that exist in the physical space also being extended into the virtual space, with similar logic and mechanism.

This dissertation is intended to consider the virtual space as part of (extended) urban space, which as well as the physical space, plays its roles as the medium of domination and arena of contestation. The culture phenomena of \moha\ that suddenly became popular reflects the rapid change experienced by the country in the early 21st century. This unique case provide an interesting aspect to view the Chinese society, people's reaction to the censorship and surveillance, the subordinate's tactic for political expressions, and the flexibility of language and meme.

\section{Methodology}

Except this chapter of introduction, this dissertation contains two main chapters, followed by a conclusion. Chapter \ref{lit} builds the theoretical framework, based on literatures in the fields of urban design, politics, and media study, two key hypotheses will be discussed and framed: the translations between explicit, direct political expressions and ordinary, implicit ones; and the virtual space as the extension of physical space.

Chapter \ref{case} will be a case study of the subculture of \moha, where I investigated the history of Chinese Internet and relevant policies and regulations, which are the base of the contemporary Internet censorship in China; then, the origin and materials used in the \moha\ subculture---Jiang Zemin's activities, related videos and text materials, etc., will be discussed, followed by a review of the subculture itself, how and by whom the memes relevant contents are being created and circulated. With the theoretical framework established in the previous chapter, the mechanism, and the possible reason and intention behind this subculture will be discussed.

Due to the inaccurate nature of the online information, this dissertation has choosen to use multiple types of sources as research materials. While building the theoretical framework, academic papers and books will be referenced to ensure the theory to be solid and reliable. In the case study, while refer to the censorship and surveillance by the state, original regulations and policies will be cited as they are basis of the government administration. Open source data will be used to analyze the confirmable facts---Internet population, web traffic, etc. For facts related to the \moha\ subculture, because it has never been (nor been allowed to) studied in mainland China, there are limited reliable data sources available, therefore, relatively ``informal'' sources will be used: personal blogs, private news sites, websites with political biases, etc. These sources might include exaggerated information, hearsays, rumours that hard or nearly impossible to verify. Since the focus of this dissertation is on regular Internet users, for whom what information is most widely perceived is not dependent on the reliability, but the accessibility; and they will react on these information assuming they are true; thus these informal information sources will be also important as part of the knowledge of regular Internet users. Therefore, the formal and informal sources will be equally discussed in the case study, while the meta-information of sources will be noted.
