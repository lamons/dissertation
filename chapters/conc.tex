\chapter{Conclusion}

\dcap{\textit{M}}{\textit{\'o-h\'a} has been} a unique cultural phenomena, not only in China but in the world. Unlike other Internet memes, the memes about Jiang has an unusually long history that more than 15 years. In early 2000s, the spiritual practice group Falun Gong started to release a large amount of materials of the negative images of the then-president Jiang Zemin. Since the Chinese Internet censorship started from 1990s, most of these materials were blocked for regular Chinese Internet users, and only available from overseas or by using proxy software. However, in early 2010s, while after the presidency of the current president Xi Jinping, these materials were used to create a large amount of memes and derivatives, and circulated on the Internet virally since 2014. At the beginning most participants were high-educated groups or tech savvy groups whom have related knowledge and/or could access to proxy software, but along with the subculture of \moha\ (toad-worship) was initiated and became popular, the phenomena involves people from various social groups.

The cause of the phenomena include many factors. A series of culture and information related policies passed during Xi's presidency emphasize a heavy censorship including real-name system, content filtering, restricting foreign contents, tightening the block of proxy software, etc.\ in the culture and Internet scene. The state has been trying to control the domestic Internet as part of its territory, as well as its geographical land, by making the  metaphorical border in order to control the circulation of information, and using technical means to match the online identity with actual person. The discipline was practiced in a way that its political intentions are covered by the technical details, and can hardly be noticed by regular users. The behaviour of regular Internet users is trained and regulated through minor adjustments, creating a sense of insecurity of talking about dangerous, mostly politically sensitive content, and thus the self-censorship is achieved. These pressing regulation raised provoke sentiments among the Internet users. Due to the heavy censorship for the expression criticizing political subjects (policies, the government or the \cpc\ themselves, etc.), and the increasing cost (economical, technical, and political) to access the uncensored Internet---some people have been arrested by posting content that unfavorable for the government, people started searching for alternative way for their political expression. Toad-worshippers use extremely ordinary languages, seemingly irrelevant quotes, minimalist symbols, and everyday objects which can hardly be censored, to trigger and refer to the materials about Jiang, which shared among participants of the subculture. Consider the virtual space as the public space, \moha\ could be understood as a way of using creative way to \textit{loosen} the space, to challenge the heavily regulated order of the public space and create an insurgent space to claim their own demand. People who share the collective knowledge and political position that giving the ability to understand the memes creates the context of \moha. In this shared ``space'', the original meaning of the everyday languages and symbols are hijacked, and become the signal and material of the ritual that allow subordinate people to challenge the discipline established by the state by simply \textit{talking} about politics---by insinuating, even making fun of the former top party leader.
% While the context of \moha---created by people who share the collective knowledge and political position that can understand the memes---emerges, the original meaning of the everyday languages and symbols are hijacked, and become the signal and material of the ritual that allow subordinate people to challenge the discipline established by the state by simply \textit{talking} about politics---by referring to, even making fun of the former top party leader.

The fever of \moha\ provides a valuable opportunity to unpack the complicated contemporary Chinese informal politics, and then the general mechanism of grassroot political discourse and the generation and circulation of Internet memes. Future studies about this topic could be taken: quantitative data is still lacking in the current research, and depth study about the participants and the socio-cultural effect of this subculture are still needed.



