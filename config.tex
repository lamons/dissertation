\hypersetup{colorlinks=false,
			linktocpage,
			% pdfborder={0 0 1},
} % Comment this line if you don't wish to have colored links

\makeatletter
\renewcommand*\l@figure{\@dottedtocline{1}{1.5em}{2.3em}}
\makeatother

\usepackage{microtype} % Improves character and word spacing
% \usepackage{zhmCJK}
% \setCJKmainfont{fonts/FandolSong-Regular.ttf}[
	% BoldFont = fonts/FandolSong-Bold.ttf,
	% ItalicFont = fonts/FandolKai-Regular.ttf
% ]
\usepackage{xeCJK}
\setCJKmainfont{Source Han Serif SC}[
	BoldFont = Source Han Serif SC Bold,
	ItalicFont = FZNewKai-Z03
]
\setmonofont{Bitstream Vera Sans Mono}[Scale=MatchLowercase]
\usepackage{fontspec}
% \usepackage{polyglossia}
% \usepackage[nonoverlap]{ruby}
% \renewcommand\rubysize{0.6}
\renewcommand\smallcaps[1]{%
	{\addfontfeature{LetterSpace=5.0}\textsc{\MakeLowercase{#1}}}}

\setmainfont{Garamond Premier Pro}[
  Numbers=OldStyle,
  ItalicFont=Garamond Premier Pro Italic,
  % ItalicFeatures={
  % SizeFeatures={
	  % {Size={12-}, Ligatures={Common, Rare}}
  % }},
  Ligatures={Common},
]%%
\newfontfamily\headingfont[
  Ligatures={Common,Rare,Historic},
  ItalicFont=Garamond Premier Pro Italic Display,
  Numbers=OldStyle,
]{Garamond Premier Pro Display}
\newfontfamily\fancyfont[
  Ligatures={Common,Rare,Historic},
  ItalicFont=Garamond Premier Pro Italic,
  Numbers=OldStyle,
]{Garamond Premier Pro}
\newfontfamily\secfont[
  % Ligatures={Common,Rare,Historic},
  ItalicFont=Garamond Premier Pro Semibold Italic Caption,
  Numbers=OldStyle,
]{Garamond Premier Pro Semibold Caption}

% \newlength\TitleOverhang
% \setlength\TitleOverhang{.6cm}

% \newcommand\Overhang[1]{%
  % \llap{\makebox[\TitleOverhang][l]{#1}}%
% }
% \titleformat{\section}
  % {\fancyfont\Large\itshape}{\Overhang\thesection}{0em}{}
\usepackage[inline]{enumitem}
\usepackage{xpinyin}
% \setitemize{noitemsep,topsep=0pt,parsep=0pt,partopsep=0pt}
\setlist{noitemsep,topsep=0pt,parsep=0pt,partopsep=0pt}
\usepackage[parfill]{parskip}
% \setlength{\parindent}{0px}
\makeatletter
% Paragraph indentation and separation for normal text
\renewcommand{\@tufte@reset@par}{%
  \setlength{\RaggedRightParindent}{0.0pc}%
  \setlength{\JustifyingParindent}{0.0pc}%
  \setlength{\parindent}{0pc}%
  \setlength{\parskip}{\baselineskip}%
}
\@tufte@reset@par
\makeatother

\makeatletter
\newcommand\chapterauthor[1]{#1\gdef\@chapterauthor{#1}}
\def\@chapterauthor{}
\fancypagestyle{mystyle}{%
\fancyhf{}%
\renewcommand{\chaptermark}[1]{\markboth{##1}{}}%
\fancyhead[LE]{\thepage\quad\smallcaps{\newlinetospace{\leftmark}}}%
\fancyhead[RO]{\smallcaps{\newlinetospace{\@chapterauthor}}\quad\thepage}%
}
\makeatother

% \usepackage[T1]{fontenc}
% \usepackage[utf8]{inputenc}
\usepackage[greek,russian,english]{babel}

\usepackage{expex}
\makeatletter
\renewcommand{\exnoprint}{\llap{\ep@exnoformat{\ep@rawexnoprint}}}
\makeatother
\usepackage{titletoc,mdframed,pdfpages,soul,varioref,pgfplots}
\usepackage[acronym]{glossaries}
\pgfplotsset{
compat=newest,
ticklabel style = {font=\footnotesize},
grid style={dotted,black},
}
\usetikzlibrary{pgfplots.dateplot}
\usepgfplotslibrary{dateplot}
  % \usepackage{etoolbox}
  \setcounter{tocdepth}{1}
  \pretocmd{\tableofcontents}{\begin{mdframed}[outermargin=\dimexpr-\marginparwidth-\marginparsep\relax,innermargin=0pt,hidealllines=true]\let\cleardoublepage\relax}{}{}
  \apptocmd{\tableofcontents}{\end{mdframed}}{}{}

\usepackage[backend=bibtex,autocite=footnote,style=authoryear-icomp,natbib=true,dashed=false]{biblatex}

\defbibheading{bibliography}[\bibname]{%
  \chapter*{#1}%
  \markboth{#1}{#1}}

\usepackage{lipsum,lettrine,type1cm} % Inserts dummy text
\newcommand{\dcap}[2]{\lettrine[lines=2, findent=3pt, nindent=0pt]{#1}{}{\smallcaps{#2}}}

\usepackage{booktabs} % Better horizontal rules in tables

\usepackage{graphicx} % Needed to insert images into the document
\graphicspath{{graphics/}} % Sets the default location of pictures
\setkeys{Gin}{width=\linewidth,totalheight=\textheight,keepaspectratio} % Improves figure scaling

\usepackage{fancyvrb} % Allows customization of verbatim environments
\fvset{fontsize=\normalsize} % The font size of all verbatim text can be changed here

\newcommand{\hangp}[1]{\makebox[0pt][r]{(}#1\makebox[0pt][l]{)}} % New command to create parentheses around text in tables which take up no horizontal space - this improves column spacing
\newcommand{\hangstar}{\makebox[0pt][l]{*}} % New command to create asterisks in tables which take up no horizontal space - this improves column spacing

\usepackage{xspace} % Used for printing a trailing space better than using a tilde (~) using the \xspace command

\newcommand{\monthyear}{\ifcase\month\or January\or February\or March\or April\or May\or June\or July\or August\or September\or October\or November\or December\fi\space\number\year} % A command to print the current month and year

\newcommand{\openepigraph}[2]{ % This block sets up a command for printing an epigraph with 2 arguments - the quote and the author
\begin{fullwidth}
\sffamily\large
\begin{doublespace}
\noindent\allcaps{#1}\\ % The quote
\noindent\allcaps{#2} % The author
\end{doublespace}
\end{fullwidth}
}

\newcommand{\blankpage}{\newpage\hbox{}\thispagestyle{empty}\newpage} % Command to insert a blank page

\usepackage{units} % Used for printing standard units

\newcommand{\hlred}[1]{\textcolor{Maroon}{#1}} % Print text in maroon
\newcommand{\hangleft}[1]{\makebox[0pt][r]{#1}} % Used for printing commands in the index, moves the slash left so the command name aligns with the rest of the text in the index
\newcommand{\hairsp}{\hspace{1pt}} % Command to print a very short space
\newcommand{\ie}{\textit{i.\hairsp{}e.}\xspace} % Command to print i.e.
\newcommand{\eg}{\textit{e.\hairsp{}g.}\xspace} % Command to print e.g.
\newcommand{\na}{\quad--} % Used in tables for N/A cells
\newcommand{\measure}[3]{#1/#2$\times$\unit[#3]{pc}} % Typesets the font size, leading, and measure in the form of: 10/12x26 pc.
\newcommand{\tuftebs}{\symbol{'134}} % Command to print a backslash in tt type in OT1/T1

\providecommand{\XeLaTeX}{X\lower.5ex\hbox{\kern-0.15em\reflectbox{E}}\kern-0.1em\LaTeX}
\newcommand{\tXeLaTeX}{\XeLaTeX\index{XeLaTeX@\protect\XeLaTeX}} % Command to print the XeLaTeX logo while simultaneously adding the position to the index

\newcommand{\doccmdnoindex}[2][]{\texttt{\tuftebs#2}} % Command to print a command in texttt with a backslash of tt type without inserting the command into the index

\newcommand{\doccmddef}[2][]{\hlred{\texttt{\tuftebs#2}}\label{cmd:#2}\ifthenelse{\isempty{#1}} % Command to define a command in red and add it to the index
{ % If no package is specified, add the command to the index
\index{#2 command@\protect\hangleft{\texttt{\tuftebs}}\texttt{#2}}% Command name
}
{ % If a package is also specified as a second argument, add the command and package to the index
\index{#2 command@\protect\hangleft{\texttt{\tuftebs}}\texttt{#2} (\texttt{#1} package)}% Command name
\index{#1 package@\texttt{#1} package}\index{packages!#1@\texttt{#1}}% Package name
}}

\newcommand{\doccmd}[2][]{% Command to define a command and add it to the index
\texttt{\tuftebs#2}%
\ifthenelse{\isempty{#1}}% If no package is specified, add the command to the index
{%
\index{#2 command@\protect\hangleft{\texttt{\tuftebs}}\texttt{#2}}% Command name
}
{%
\index{#2 command@\protect\hangleft{\texttt{\tuftebs}}\texttt{#2} (\texttt{#1} package)}% Command name
\index{#1 package@\texttt{#1} package}\index{packages!#1@\texttt{#1}}% Package name
}}

% A bunch of new commands to print commands, arguments, environments, classes, etc within the text using the correct formatting
\newcommand{\docopt}[1]{\ensuremath{\langle}\textrm{\textit{#1}}\ensuremath{\rangle}}
\newcommand{\docarg}[1]{\textrm{\textit{#1}}}
\newenvironment{docspec}{\begin{quotation}\ttfamily\parskip0pt\parindent0pt\ignorespaces}{\end{quotation}}
\newcommand{\docenv}[1]{\texttt{#1}\index{#1 environment@\texttt{#1} environment}\index{environments!#1@\texttt{#1}}}
\newcommand{\docenvdef}[1]{\hlred{\texttt{#1}}\label{env:#1}\index{#1 environment@\texttt{#1} environment}\index{environments!#1@\texttt{#1}}}
\newcommand{\docpkg}[1]{\texttt{#1}\index{#1 package@\texttt{#1} package}\index{packages!#1@\texttt{#1}}}
\newcommand{\doccls}[1]{\texttt{#1}}
\newcommand{\docclsopt}[1]{\texttt{#1}\index{#1 class option@\texttt{#1} class option}\index{class options!#1@\texttt{#1}}}
\newcommand{\docclsoptdef}[1]{\hlred{\texttt{#1}}\label{clsopt:#1}\index{#1 class option@\texttt{#1} class option}\index{class options!#1@\texttt{#1}}}
\newcommand{\docmsg}[2]{\bigskip\begin{fullwidth}\noindent\ttfamily#1\end{fullwidth}\medskip\par\noindent#2}
\newcommand{\docfilehook}[2]{\texttt{#1}\index{file hooks!#2}\index{#1@\texttt{#1}}}
\newcommand{\doccounter}[1]{\texttt{#1}\index{#1 counter@\texttt{#1} counter}}

\usepackage{makeidx} % Used to generate the index
% \makeindex % Generate the index which is printed at the end of the document

\makeatletter
\renewcommand{\@chapapp}{}% Not necessary...
\newenvironment{chapquote}[2][2em]
  {\setlength{\@tempdima}{#1}%
   \def\chapquote@author{#2}%
   \parshape 1 \@tempdima \dimexpr\textwidth-2\@tempdima\relax%
   \itshape}
  {\par\normalfont\hfill---\ \chapquote@author\hspace*{\@tempdima}\par\bigskip}
\makeatother

\newcommand{\cann}{[\smallcaps{can}]}


